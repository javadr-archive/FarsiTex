\documentclass[a4paper]{article}
%\usepackage[top=2cm,left=2cm]{geometry}
\usepackage{verbatim}
\usepackage{pstricks}
\usepackage{xepersian}
\settextfont[Scale=1]{XB Zar}
\setlatintextfont[Scale=1]{Linux Libertine}
\title{ راهنمای نصب فارسی‌تِک بر روی \lr{MiK\TeX{}2.4}}
\author{سیّد محمّد جواد رضویان\\
دانشگاه قم\\
گروه علوم‌کامپیوتر}
%\university{دانشگاه قم}
\date{\today}
\begin{document}
\maketitle
%\baselineskip=.91cm
با توجه به اینکه فارسی‌تک مبتنی بر \lr{\LaTeX{}2.09} است لذا باید روی \lr{Mik\TeX{}2.4} یا نسخه‌های قبل از آن نصب گردد\footnote{البته با کمی جستجو در اینترنت می‌توانید به نحوه نصب روی نسخه‌های بالاتر میک‌تک پی ببرید.}. توضیحاتی که در ذیل می‌آید مبتنی بر فایل‌هایی است که در سی‌دی حاوی این راهنما قرار دارد و هیچ تضمینی برای اجرا و عملکرد صحیح آن بر روی دیگر  توزیع‌های فارسی‌تک نمی‌باشد و تمام مطالب زیر صرفاً یک تجربه شخصی در نصب فارسی‌تک می‌باشد. \textbf{پیش از شروع به نصب هر نسخه‌ای که قبلاً از میک‌تک یا فارسی‌تک روی سیستمتان نصب شده را \lr{uninstall} نمایید.} گام‌هایی که ذیلاً  شرح داده می‌شود را به ترتیب اجرا نمایید.
%\set{inum}{0}
\begin{enumerate}
	\item 
	\textbf{ نصب \lr{Mik\TeX{}2.4}.}	 (توجه: در غالب گام‌های زیر شما تنها کافی است که گزینه \lr{Next} را انتخاب نمایید، ارائه این مراحل تنها برای اطمینان شما از روند صحیح نصب آورده شده است.)
		\begin{enumerate}
			\item 
		از پرونده  \lr{Mik\TeX{}2.4} ‬فایل ‪ \lr{setup-2.4.1661.exe}‬را اجرا کنید.‬
			\item	
		در پنجره \lr{Welcone to the MiKTeX Setup Wizard} گزینه \lr{Next} را انتخاب کنید.
			\item
		از پنجره \lr{Setup Task} گزینه \lr{Install} را انتخاب کنید.
			\item
		از پنجره \lr{Package Set} گزینه \lr{Total} را انتخاب کنید.
			\item
		از پنجره \lr{Shared vs. Private Installation} گزینه \lr{Install MiKTeX for everyone} را انتخاب کنید.
			\item
		در پنجره \lr{Local Package Repository} در کادر  \lr{Path to local package repository} آدرس پرونده \lr{Mik\TeX{}2.4} را از روی سی‌دی انتخاب کنید.
\linebreak \underline{توجه داشته باشید که مسیر پیش‌فرض همان مسیر مطلوب است.}
			\item
		در پنجره \lr{Installation Folder} در کادر  \lr{Path to installation folder} آدرس \lr{C:$\backslash{}$texmf} را انتخاب کنید.
			\item 
		در پنجره \lr{Program Folder} نام شاخه‌ برنامه که در منوی \lr{start} نصب خواهد شد را بنویسید.
			\item 
		از پنجره \lr{Local TEXMF Tree} گزینه \lr{Create local TEXMF tree} را انتخاب کنید و مسیر مشخص شده در کادر \lr{Path to root folder} را بدون تغییر همان مسیر \lr{C:$\backslash{}$localtexmf} قرار دهید.
			\item 
		از پنجره \lr{Additional TEXMF Folder Trees} گزینه \lr{Don't incorporate existing TEXMF folder trees now} را انتخاب کنید.
			\item 
		در پنجره \lr{Setup Information} گزینه \lr{Next} را انتخاب کنید.
			\item
		هم‌اکنون نصب میک‌تک آغاز می‌شود کمی صبور باشید $\cdots$ 
		\end{enumerate}
	\item 
	\textbf{نصب \lr{GhostScript}.} از پرونده \lr{GhostScript} فایل \lr{gs700w32.exe} را اجرا کنید.
	\item 
	\textbf{نصب \lr{GSview}.}
		\begin{enumerate}
			\item 
		از پرونده \lr{GSview} فایل \lr{gsv40w32.exe} را اجرا کنید.
			\item 
		پس از نصب برنامه را اجرا کنید. اگر قبلاً برنامه را نصب و \lr{register} کرده باشید به انجام گام‌های زیر نیازی نیست.
			\item 
		به منوی \lr{Help} بروید و \lr{Register$\cdots$} را انتخاب کنید تا پنجره \lr{GSview Registration} باز شود. به طور پیش فرض به هنگام اجرای برنامه در صورتی که برنامه  \lr{register} نشده باشد این پنجره را در پیش رو خواهید داشت.
			\item 
		از	\lr{GSview Registration}دکمه \lr{Register Now} را انتخاب کنید.
			\item 
		اطلاعات خواسته شده در کادرهای \lr{Registered to:} و \lr{Number} را از فایل \lr{Register gsv40w32.txt} کپی پیست نمایید و سپس دکمه \lr{OK} را بزنید.
		\end{enumerate}
	\item 
	\textbf{نصب فارسی‌تک نسخه \lr{1.0pre1}.}
\textit{یکی از قابلیت‌هایی که به این نسخه افزوده شده است تولید مستقیم خروجی \lr{pdf} از فایل منبع متن است. در ادیتور فارسی‌تک پس از نصب می‌توانید با فشردن دکمه‌های \lr{Ctrl+F5} مستقیما بدون تولید فایل \lr{dvi} و \lr{ps} فایل پی‌دی‌اف متن خود را بدست آورید.
برای آشنایی با قابلیت‌های دیگر این نسخه می‌توانید به فایل راهنما که در سی‌دی قرار دارد مراجعه نمایید.}
		\begin{enumerate}
			\item 
			از پرونده \lr{FarsiTeX 1.0pre1} فایل  \lr{FarsiTeX-1.0pre1-setup.exe} را اجرا کنید.
			\item 
			در پنجره \lr{Welcome to the InstallShield Wizard for FarsiTeX} گزینه \lr{Next} را انتخاب کنید.
			\item 
			از پنجره \lr{License Agreement} گزینه \lr{I accpet the terms in the license agreement} را انتخاب کنید.
			\item 
			از پنجره \lr{Destination Folder} مسیر نصب باید \lr{C:$\backslash$localtexmf} باشد.
			\item 
			از پنجره \lr{Setup Type} گزینه \lr{Custom} را انتخاب کنید.
			\item 
			از پنجره \lr{Custom Setup} گزینه \lr{Next} را انتخاب کنید.
			\item 
			از پنجره \lr{Ready to Install the Program} گزینه \lr{Install} را انتخاب کنید.
			\item 
			از پنجره \lr{InstallSheild Wizard Completed} تیک گزینه  \lr{Lunch the program} را بردارید و سپس \lr{Finish} را بزنید.
			\item 
			حال پرونده \lr{bin} را از داخل پرونده  \lr{FarsiTeX 1.0pre1} کپی کرده و آن را در مسیر  \lr{C:$\backslash$localtexmf$\backslash$miktex} پیست نمایید. ویندوز پنجره \lr{Confirm Folder Replace} را نشان می‌دهد که باید گزینه \lr{Yes to All}  را انتخاب کنید.
		\end{enumerate}
\end{enumerate}
\begin{center}
\fcolorbox{black}{gray}{
{\begin{minipage}{.8\textwidth}
با به پایان رسیدن نصب فارسی‌تک می‌توانید از تایپ متون خود لذت ببرید. باز هم تاکید می‌گردد که هیچ تضمینی برای اجرا و نصب دیگر نسخه‌های فارسی‌تک نیست علاوه بر این در نصب همین نسخه نیز تضمینی نیست.
\end{minipage}}}
\end{center}
\end{document}
