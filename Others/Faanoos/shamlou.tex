\documentclass{article}
\usepackage[farsi]{faanoos}

\title{pas aan-gaah zamin be sokhan dar-aamad\dots}
\author{aHmad shaamlou}
\date{taabestaan-haa-ye 1343 va 1363}

\parindent=0cm
\parskip=.4em

\renewcommand{\*}{%
	\begin{center}*\end{center}
}

\begin{document}
\maketitle

pas aan-gaah zamin be sokhan dar-aamad,

va aadami, khaste va tanhaa va andish-naak bar sar-e sangi
neshaste bood, pashimaan az kerd-o-kaar-e xish

va zamin-e be-sokhan-dar-aamade baa oo chenin mi-goft:

--- be to naan daadam man, va 'alaf be goosfandaan va be gaavaan-e to,
va barg-haa-ye naazok-e tarre ke ghaategh-e naan koni.

ensaan goft: --- mi-daanam.

pas zamin goft: --- be hargoone Sedaa man baa to be sokhan dar-aamadam:
baa nasim va baad, va baa jooshidan-e cheshme-haa az sang, va baa
rizesh-e aab-shaaraan; va baa foroo-qaltidan-e bahmanaan az kooh aan-gaah
ke sakht bi-khabarat mi-yaaftam. va be koos-e tondar va Torghe-ye
toofaan.

ensaan goft: mi-daanam, mi-daanam, ammaa chegoone mi-tavaanestam raaz-e
payaam-e to raa dar-yaabam?

pas zamin baa oo, baa ensaan, chenin goft:

--- na xod in sahl bood, ke payaam-gozaaraan niz andak naboodand.

to mi-daanesti ke manat be parastandegi 'aashegham. niz na be goone-ye
'aasheghi bakht-yaar, ke zar-kharide-vaar kanizakaki baraa-ye to boodam
be ra`y-e xish. ke to raa chandaan doost mi-daashtam ke cheon dast bar
man mi-goshoodi tan va jaanam be hezaar naqme-ye xosh javaab-goo-ye
to mi-shod. ham-cheon now-'aroosi dar rakht-e zefaaf, ke naale-haa-ye
tan-aazordegi-yash be taraane-ye kashf va kaamyaari badal shavad, yaa
changi ke har zakhme raa be zir o bami del-padhir digar-goone javaabi
gooyad. --- aay, che 'aroosi ke har baar sar-be-mohr, baa bestar-e to
dar-aamad! (chonin mi-goft zamin.) dar kodaamin baadiye chaahi kardi ke
be aabi govaaraa kaam-yaabat nakardam? kojaa be dastaan-e
khoshoonat-baari ke enteZaar-e soozaan-e navaazesh-e HaaSel-khizash
baa man ast gaav-aahan dar man nahaadi ke khermani por-baar paadaash
nadaadam?

ensaan digar-baare goft: --- raaz-e payaamat raa ammaa chegoone
mi-tavaanestam dar-yaabam?

--- mi-daanesti ke manat 'aasheghaane doost mi-daaram (zamin be paasokh
goft.) mi-daanesti. va to raa man peyqaam kardam az pas-e peyqaam be
hezaar aavaa, ke del az aasemaan bar-daar, ke vaHy az khaak mi-resad.
peyqaamat kardam az pas-e peyqaam, ke moghaam-e to jaay-gaah-e
bandegaan nist, ke dar in gostare shahriyaari to; va aan-che to raa be
shahriyaari bardaasht, na 'enaayat-e aasemaan, ke mehr-e zamin ast. ---
aah ke maraa dar martebat-e khaaksaari-ye 'aasheghaane, bar gostare-ye
naamotenaahi-ye keyhaan xosh salTanati bood, ke sarsabz va aabaad az
ghodrat-haa-ye jaadooyi-e to boodam, az aan pish-tar ke to paadeshaah-e
jaan-e man be kharbandegi-e aasemaan dast-haa bar sine va pishaani be
khaak bar-nahi va maraa chenin be xaari dar-afkani.

ensaan, andish-naak va khaste va sharm-saar, az zharfaa-haa-ye dard
naale-yi kard. va zamin ham az aan-goone dar sokhan bood:

--- be-tamaami az aan-e to boodam va taslim-e to, cheon chaar-divaari-e
khaane-ye koochaki.

to raa 'eshgh-e man aan maaye tavaanaayi daad ke bar hame sar shavi.
dariqaa, pendaari gonaah-e man hame aan bood ke zir-e paa-ye to
boodam!

taa az khoon-e man parvarde shavi, be dard-mandi dandaan bar jegar
feshordam, ham-cheon maadari ke dard-e makide-shodan raa taa nowzaade-ye
daaman-e xod raa az 'oSaare-ye jaan-e xish nooshaaki dahad.

to raa aamookhtam man ke be jost-o-joo-ye sang-e aahan va rooy,
sine-ye 'aashegham raa bar-dari. va in hame az baraa-ye aan bood taa to
raa dar navaazesh-e por khoshoonat-i ke az dastaanat chashm daashtam,
afzaari be dast daade-baasham. ammaa to rooy az man bar-taafti, ke
aahan va mes raa az sang-paare koshande-tar yaafti, ke haabil raa dar
khoon keshide bood. va khaak raa az ghorbaaniyaan-e bad-koneshi-haa-ye
xish baarvar kardi.

aah, zamin-e tanhaa maande! zamin-e rahaa shode baa tanhaayi-e xish!

ensaan zir-e lab goft: --- taghdir chonin bood. magar aasemaan
ghorbaani-yi mi-xaast.

--- na, ke maraa goorestaan-i mi-khaahad! (chonin goft zamin).

va to bi eHsaas-e 'amigh-e sarshekastegi, chegoone az <<taghdir>> sokhan
mi-gooyi ke joz bahaane-ye taslim-e bi-hemmataan nist?

aan afsoon-kaar be to mi-aamoozad ke 'edaalat az 'eshgh vaalaa-tar ast.
--- dariqaa ke agar 'eshgh be kaar mi-bood, hargez setami dar vojood
nemi-aamad taa be 'edaalati naa-bekaaraane az aan dast niaazi padid
oftad. --- aan-gaah chashmaan-e to raa bar-baste, shamshiri dar kafat
mi-godhaarad, ham az aahani ke man be to daadam taa tiqe-ye
gaav-aahan koni!

inak goorestaan-i ke aasemaan az 'edaalat saakhte ast!

dariqaa viraan-e bi-HaaSeli ke manam!

\*

shab o baaraan dar viraane-haa be goft-o-goo boodand ke baad
dar-resid, miaane-be-ham-zan o por-hayaahoo.

diri nagodhasht ke khalaaf dar ishaan oftaad va qowqaa baalaa gereft
bar saraasar-e khaak, va be khaamoosh-baash-haa-ye por-qariv-e tondar
Hormat nagodhaashtand.

\*

zamin goft: --- aknoon be do-raahe-ye tafrigh reside-im.

to raa joz zard-rooyi-keshidan az bi-HaaSeli-e xish gozir nist; pas
aknoon ke be taghdir-e farib-kaar gardan nahaade-i, mardaane baash!

ammaa maraa ke viraan-e to-am, hanooz dar in madaar-e sard kaar be
paayaan nareside-ast:

ham-cheon zani 'aashegh ke be bestar-e ma'shoogh-e az-dast-rafte-ye
xish mi-khazad taa boo-ye oo raa dar-yaabad, saal-hame-saal be
moghaam-e nokhostin baaz mi-aayam baa ashk-haa-ye khaaTere.

yaad-e bahaaraan bar man forood mi-aayad bi-aan-ke az shokhmi taaze
baar bar-gerefte-baasham va gostaresh-e rishe-yi raa dar baTn-e khod
eHsaas konam; va abr-haa baa khas-o-khaari ke dar aaqoosham
xaahand nahaad, baa ashk-haa-ye 'aghim-e xish be tasallaa-yam
xaahand kooshid.

jaan-e maraa ammaa tasallaayi moghaddar nist:

be qiyaab-e dard-naak-e to solTaan-e shekaste-ye kahkeshaan-haa
xaaham-andishid, ke be afsoon-e palidi az paay dar-aamadi;
va radd-e angoshtaanat raa
\vspace{-.8\baselineskip}
\begin{tabbing}
bar tan-e nowmid-e xish\=\\
			\>dar khaaTere-i geryaan\\
			\>jost-o-joo\\
			\>xaaham kard.\\
\end{tabbing}
\end{document}
