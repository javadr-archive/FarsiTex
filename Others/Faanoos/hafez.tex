\begin{oldpoem}
الا يا ايها الساقی ادر کاسا و ناولها&
که عشق آسان نمود اول ولی افتاد مشکل‌ها\\
به بوی نافه‌ای کاخر صبا زان طره بگشايد&
ز تاب جعد مشکينش چه خون افتاد در دل‌ها\\
مرا در منزل جانان چه امن عيش چون هر دم&
جرس فرياد می‌دارد که بربنديد محمل‌ها\\
به می سجاده رنگين کن گرت پير مغان گويد&
که سالک بی‌خبر نبود ز راه و رسم منزل‌ها\\
شب تاريک و بيم موج و گردابی چنين هايل&
کجا دانند حال ما سبکباران ساحل‌ها\\
همه کارم ز خود کامی به بدنامی کشيد آخر&
نهان کی ماند آن رازی کز او سازند محفل‌ها\\
حضوری گر همی‌خواهی از او غايب مشو حافظ&
متی ما تلق من تهوی دع الدنيا و اهملها\\
\\
الا يا ايها الساقی ادر کاسا و ناولها&
که عشق آسان نمود اول ولی افتاد مشکل‌ها\\
به بوی نافه‌ای کاخر صبا زان طره بگشايد&
ز تاب جعد مشکينش چه خون افتاد در دل‌ها\\
مرا در منزل جانان چه امن عيش چون هر دم&
جرس فرياد می‌دارد که بربنديد محمل‌ها\\
به می سجاده رنگين کن گرت پير مغان گويد&
که سالک بی‌خبر نبود ز راه و رسم منزل‌ها\\
شب تاريک و بيم موج و گردابی چنين هايل&
کجا دانند حال ما سبکباران ساحل‌ها\\
همه کارم ز خود کامی به بدنامی کشيد آخر&
نهان کی ماند آن رازی کز او سازند محفل‌ها\\
حضوری گر همی‌خواهی از او غايب مشو حافظ&
متی ما تلق من تهوی دع الدنيا و اهملها\\
الا يا ايها الساقی ادر کاسا و ناولها&
که عشق آسان نمود اول ولی افتاد مشکل‌ها\\
به بوی نافه‌ای کاخر صبا زان طره بگشايد&
ز تاب جعد مشکينش چه خون افتاد در دل‌ها\\
مرا در منزل جانان چه امن عيش چون هر دم&
جرس فرياد می‌دارد که بربنديد محمل‌ها\\
به می سجاده رنگين کن گرت پير مغان گويد&
که سالک بی‌خبر نبود ز راه و رسم منزل‌ها\\
شب تاريک و بيم موج و گردابی چنين هايل&
کجا دانند حال ما سبکباران ساحل‌ها\\
همه کارم ز خود کامی به بدنامی کشيد آخر&
نهان کی ماند آن رازی کز او سازند محفل‌ها\\
حضوری گر همی‌خواهی از او غايب مشو حافظ&
متی ما تلق من تهوی دع الدنيا و اهملها\\
الا يا ايها الساقی ادر کاسا و ناولها&
که عشق آسان نمود اول ولی افتاد مشکل‌ها\\
به بوی نافه‌ای کاخر صبا زان طره بگشايد&
ز تاب جعد مشکينش چه خون افتاد در دل‌ها\\
مرا در منزل جانان چه امن عيش چون هر دم&
جرس فرياد می‌دارد که بربنديد محمل‌ها\\
به می سجاده رنگين کن گرت پير مغان گويد&
که سالک بی‌خبر نبود ز راه و رسم منزل‌ها\\
شب تاريک و بيم موج و گردابی چنين هايل&
کجا دانند حال ما سبکباران ساحل‌ها\\
همه کارم ز خود کامی به بدنامی کشيد آخر&
نهان کی ماند آن رازی کز او سازند محفل‌ها\\
حضوری گر همی‌خواهی از او غايب مشو حافظ&
متی ما تلق من تهوی دع الدنيا و اهملها\\
الا يا ايها الساقی ادر کاسا و ناولها&
که عشق آسان نمود اول ولی افتاد مشکل‌ها\\
به بوی نافه‌ای کاخر صبا زان طره بگشايد&
ز تاب جعد مشکينش چه خون افتاد در دل‌ها\\
مرا در منزل جانان چه امن عيش چون هر دم&
جرس فرياد می‌دارد که بربنديد محمل‌ها\\
به می سجاده رنگين کن گرت پير مغان گويد&
که سالک بی‌خبر نبود ز راه و رسم منزل‌ها\\
شب تاريک و بيم موج و گردابی چنين هايل&
کجا دانند حال ما سبکباران ساحل‌ها\\
همه کارم ز خود کامی به بدنامی کشيد آخر&
نهان کی ماند آن رازی کز او سازند محفل‌ها\\
حضوری گر همی‌خواهی از او غايب مشو حافظ&
متی ما تلق من تهوی دع الدنيا و اهملها\\
الا يا ايها الساقی ادر کاسا و ناولها&
که عشق آسان نمود اول ولی افتاد مشکل‌ها\\
به بوی نافه‌ای کاخر صبا زان طره بگشايد&
ز تاب جعد مشکينش چه خون افتاد در دل‌ها\\
مرا در منزل جانان چه امن عيش چون هر دم&
جرس فرياد می‌دارد که بربنديد محمل‌ها\\
به می سجاده رنگين کن گرت پير مغان گويد&
که سالک بی‌خبر نبود ز راه و رسم منزل‌ها\\
شب تاريک و بيم موج و گردابی چنين هايل&
کجا دانند حال ما سبکباران ساحل‌ها\\
همه کارم ز خود کامی به بدنامی کشيد آخر&
نهان کی ماند آن رازی کز او سازند محفل‌ها\\
حضوری گر همی‌خواهی از او غايب مشو حافظ&
متی ما تلق من تهوی دع الدنيا و اهملها\\
\end{oldpoem}
