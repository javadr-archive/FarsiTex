\documentclass{article}
\usepackage[farsi]{faanoos}

\title{taghdim be Ezme baa 'eshgh o nekbat\footnote{%
	az majmoo'e-ye del-tangi-haa-ye naghghaash-e
	khiaabaan-e chehel o hashtom}%
}
\author{jey di salinjer}
\date{}

\begin{document}

\maketitle

taazegi-haa baa post-e havaayi da'vat-naame-i baraay-e sherkat dar yek
jashn-e 'aroosi be dastam reside ke dar hejdahom-e aavril dar engelis
bargozaar mi-shavad.  ettefaaghaN baraa-ye raftan be in 'aroosi del too-ye
delam naboode, banaa-bar-in hamin ke da'vat-naame resid,
bi-aan-ke fekr-e hazine-ash raa bokonam, pish-e xod goftam baa havaapeymaa
raahi-e safar-e khaarej mi-shavam.  ammaa az hamaan vaght mowDoo' raa az
janbe-haa-ye ziaadi baa zanam, ke bi-andaaze aadam-e ma'ghooli ast, sabok
o sangin karde-am va baa ham be in natije residim ke fekrash raa az sar
biroon konam, choon be kolli faraamoosh karde boodam ke maadar-zanam cheshm
be raah ast ke do hafte-ye aakhar-e maah-e aavril raa baa maa begodharaanad.
raastash, man forSat nadaaram maamaan-e grencher raa ziaad bebinam o oo ham
'omrash raa karde ast.  Aakhar panjaah o hasht saalash ast. (xodash ham
in raa ghabool daarad.)

har-chand ke farghi nemi-konad va har jaa baasham khiaal nemi-konam az oon
aadam-haa-yi baasham ke baraa-ye bar-ham-zadan-e ezdevaaji ke aakhar o
'aaghebat-e xoshi nadaarad dast roo-ye dast mi-godhaarand.  banaa-bar-in dast
be kaar shode-am taa chand maTlab-e efshaa-garaane raa darbaare-ye 'aroos,
aan-Towr ke shesh saal pish oo raa shenaakhtam, roo-ye kaaqadh biaavaram. 
che baak agar yaad-daasht-haa-yam baraa-ye yeki do laHZe owghaat-e daamaad
raa --ke nemi-shenaasam-- talkh konad.  choon ghaSd-e man xoshHaal kardan-e
kasi nist balke raastash, ghaSd-e man bish-tar taHdhib ast, aamoozesh ast.

dar aavril-e 1944, man yeki az Hodood-e shaSt neZaami-e aamrikaayi boodam
ke dar dEvOn-e engelis, zir-e naZar-e saazmaan-e Dedd-e-jaasoosi-e aan-jaa,
yek dowre-ye aamoozesh-e Hamle raa mi-godharaandim ke kamaabish takhaSSoSi
bood.  Haalaa ke be godhashte negaah mi-konam be naZaram mi-resad ke maa shaSt
nafar az in naZar ke hich-kodaam ahl-e begoo-beshno naboodim, jam'-e nesbataN
kam-naZiri raa tashkil mi-daadim.  maa hame be Towr-e kolli ahl-e
naame-neveshtan boodim va dar khaarej az saa'at-haa-ye khedmat tanhaa Harfi
ke baa ham mi-zadim in bood ke davaat-e yek-digar raa, agar kaari baa aan
nadaashtim, mi-xaastim.  vaghti naame nemi-neveshtim yaa sar-e kelaas
naboodim, har-kodaam az maa faghaT too-ye laak-e xodemaan boodim.  man
ma'moolaN dar rooz-haa-ye aaftaabi baraa-ye didan-e gorooh-haa-ye namaayesh be
roostaa-haa-ye aTraaf mi-raftam va rooz-haa-ye baaraani yek gooshe-ye khoshk
peydaa mi-kardam o mi-neshastam ketaab mi-xaandam, aan ham dorost be
faaSele-ye yek Tool-e dast az miz-e ping-pong.

dowre-ye aamoozesh se hafte Tool keshid va dar yek rooz-e shanbe, ke
seyl-aasaa baaraan mi-baarid, tamaam shod.  dar saa'at-e haft-e aan shab-e
aakhar, afraad-e maa hame gharaar bood baa ghaTaar be landan beravim, va dar
aan-jaa, aan-Towr ke shaaye' bood, be piaade-neZaam va bakhsh-haa-ye
havaa-bord mo'arrefi shavim taa baraa-ye rooz-e aaqaaz-e 'amaliaat (rooz-e
aaqaaz-e 'amaliaat, eshaare ast be rooz-e 6 zhui`an-e 1944 ke niroo-haa-ye
mottafeghin, dar jang-e jahaani-e dovom, be faraanse yooresh bordand. m.)
eHDaar shavim.  dar saa'at-e se-ye ba'd-az-Zohr, man digar hame-ye
lavaazemam raa too-ye kif-e sarbaazi jaa daade boodam, az jomle yek kif-e
berezenti-e maask-e Dedd-e-gaaz raa ke por az ketaab-haayi bood ke az aan
soo-ye oghyaanoos aavarde boodam (maask-e Dedd-e-gaaz raa chand hafte pish az
aan, ke too-ye keshti-e mooritaanaa boodam, az rowzane biroon andaakhte
boodam, baa in ke kaamelaN mi-daanestam ke agar doshman yek-vaght gaaz be
kaar bebarad, hargez be mowghe' dastam be aan nemi-resad).  Yaadam mi-aayad
ke moddati Toolaani posht-e panjere-ye kolbe-ye ordoogaah-e-maan istaade
boodam va baaraan-e deltang-konande raa ke orib mi-baarid negaah mi-kardam
va angosht-e sabbaabe-am befahmi-nafahmi dard mi-kard.  az posht-e sar
Sedaa-ye khesh-khesh-e 'aSabaani-konande-ye ghalam-haa-ye xod-nevis-e ziaadi
raa bar SafHe-haa-ye kaaqadh-e post-e havaayi mi-shenidam.  naagahaan bi-hadaf
az posht-e panjere raftam va baaraaniam raa pooshidam, shaal-gardanam raa
bastam, gaalesh-haa-yam raa pooshidam, dastkesh-haa-ye pashmi-am raa dast
karde va kaasketam raa sar godhaashtam (kaasteki ke hanooz ke hanooz ast
baraayam dast mi-girand ke be salighe-ye xod kaj bar sar mi-godhaashtam va
andaki taa roo-ye goosh-haa-yam paayin mi-keshidam).  sepas ba'd az
mizaan-kardan-e saa'atam baa saa'at-e too-ye aabriz-gaah, ghadam-zanaan az
roo-ye tappe-ye sangi va Toolaani va khis saraazir shodam va raah-e shahr raa
dar pish gereftam.  E'tenaayi be ra'd-o-bargh, ke aTraafam raa rowshan
mi-kard, nadaashtam.  ra'd-o-bargh gah-gaah shomaare-ye roo-ye lebaasam raa
mon'akes mi-kard.

dar markaz-e shahr, ke shaayad az hame-jaa-ye shahr khis-tar bood, jelo-ye
kelisaa-yi istaadam taa taablo-e aagahi-e aan raa bexaanam, bish-tar be in
dalil ke a'daad-e barjeste-ye sefid-rang, roo-ye zamine-ye siaah tavajjoh-e
maraa jalb karde bood va andaki ham be in dalil ke pas az se saal
maandegaari too-ye artesh be xaandan-e taablo-haa-ye aagahi 'aadat karde
boodam.  roo-ye taablo neveshte shode bood ke az saa'at-e se o rob' tamrin-e
aavaaz-e daste-jam'i-e bachche-haa shoroo' mi-shavad.  ebtedaa be saa'at-e
mochi-e xod va sepas do-baare be taablo-ye aagahi negaah kardam.  roo-ye yek
barg kaaqadh ke chasbaande shode bood naam-e bachche-haa-yi ke enteZaar
mi-raft dar tamrin sherkat konand radif shode bood.  zir-e baaraan istaadam
va hame-ye esm-haa raa xaandam.  sepas paa be kelisaa godhaashtam.

davaazdah sizdah nafar in-jaa va aan-jaa roo-ye nimkat-haa neshaste boodand,
roo-ye zaanoo-ye chandin nafar az aan-haa gaalesh-haa-ye bachche-gaane be
Towr-e vaaroone dide mi-shod.  az miaan-e nimkat-haa godhashtam va dar radif-e
jelo neshastam.  too-ye jaaygaah-e korsi-e khaTaabe, nazdik be bist
bachche-ye haft taa sizdah saale, ke bish-tar dokhtar boodand, tang-e ham
dar se radif, roo-ye Sandali-haa-ye taalaar neshaste boodand.  dar aan laHZe
morabbi-e ham-soraayi-e aan-haa, zani tanoomnad baa lebaas-e pashmi-e
mardaane, be aan-haa andarz mi-daad ke vaghti aavaaz mi-xaanand
dahaaneshaan raa baaz-tar konand.  Az aan-haa porsid ke kasi taa Haalaa
daastaan-e gonjeshk-e koochaki raa shenide ke jora`t kard bi aan ke neok-e
koochakash raa baaz-e baaz-e baaz konad aavaaz bexaanad?  ZaaheraN kasi
nashnide bood, cheon hame baa negaahi tohi be oo khire shodand.  Aan vaght oo
donbaale-ye Harfash raa gereft va goft ke delash mi-xaahad hame-ye
bachche-haa-yash ma'ni-e kalame-haa-yi raa ke mi-xaanand dark konand;
na in ke, medhl-e TooTi-haa-ye zabaan-nafahm, faghaT adaa konand.  sepas noti
raa baa diaapaazon-e xod navaakht va bachche-haa, mecl-e
vazne-bardaar-haa-ye naa-baaleq, daftar-haa-ye xodeshaan raa dast
gereftand.

Aan-haa bedoon-e hamraahi baa aahang, yaa daghigh-tar gofte shavad, bedoon-e
hamaahangi baa yek-digar aavaaz mi-xaandand.  Sedaayeshaan bi aan ke
eHsaasaati baashad xosh-aahang bood taa aan jaa ke agar andaki bish-tar
tamaayolaat-e madhhabi daashtam be-saadegi be Haal-e khalse foroo mi-raftam. 
do taa az bachche-haa-ye kheyli koochak-tar gaam raa andaki mi-keshidand
ammaa albate tanhaa aahang-saaz mi-tavaanest be khaTaa-ye aan-haa pey
bebarad.  man dar 'omram in sorood-e madhhabi raa nashnide boodam ammaa delam
mi-xaast ke davaazdah sizdah saTri baashad.  chehre-ye tak-tak-e
bachche-haa raa bar-andaaz kardam va be-khoSooS yeki az aan-haa raa, ya'ni
bachche-i ke be man nazdik-tar bood va roo-ye Sandali-ye aakhar-e radif-e
jelo neshaste bood, zir-e naZar gereftam.  oo dokhtari bood ke sizdah saali
daasht, moo-haa-ye Saaf va boorash taa narme-ye goosh-haa mi-resid,
pishaani-e zibaa va cheshm-haa-ye del-faribash, be-gomaan-e man, hame-ye
chehre-haa-ye aan-jaa raa az sekke andaakhte bood.  Sedaayash aashkaaraa,
ammaa na be in dalil ke nazdik-e man bood, baa Sedaa-ye bachche-haa-ye digar
fargh daasht.  Sedaa yek parde az Sedaa-haa-ye digar baalaa-tar, shirin-tar
va moTmae`n-tar bood va xod-be-xod digaraan raa be donbaal mi-keshaand. 
ammaa in dokhtar-e javaan, be Zaaher az tavaanaayi-e xod dar sorood-xaani
yaa az in naZar ke dar chenin zamaan va makaani sorood mi-xaanad, taa
andaaze-i gerefte be naZar mi-resid; do baar miaan-e xaandan-e saTr-haa-ye
she'r motevajjeh shodam ke khamyaaze keshid.  mecl-e zan-haa khamyaaze
keshid.  baa dahaan-e baste, ammaa az naZar-e man door namaand; cheon
larzesh-e parre-haa-ye bini-ash oo raa low daad.

dar laHZe-i ke sorood tamaam shod, morabbi-e daste-ye ham-soraayaan be
eZhaar-e naZar-e Toolaani dar-baare-ye aadam-haa-yi pardaakht ke hengaam-e
meo'eZe-ye keshish nemi-tavaanand paa-haa-yeshaan raa bi-Harkat va
dahaaneshaan raa baste negah daarand.  daryaaftam ke ghesmat-e tamrin-e
aavaaz-e maraasem tamaam shode va pish az aan ke jaadoo-ye aavaaz-e
bachche-haa raa Sedaa-ye naa-hanjaar-e morabbi beshkanad boland shodam va az
kelisaa biroon aamadam.

\bigskip\noindent[edaame daarad]

\end{document}
